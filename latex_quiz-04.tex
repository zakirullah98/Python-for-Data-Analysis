\documentclass[6pt]{article}

\usepackage{amsmath,amssymb,amsfonts}
\usepackage{graphicx}
\usepackage{geometry}
\usepackage{listings}
\geometry{
 a4paper,
 total={170mm,257mm},
 left=10mm,
 top=10mm,
 right=10mm,
 }

\begin{document}

\large
\begin{center}
\noindent{\bf Name: ................SID: .................  Class-II, Quiz\#04 , Date: 24 May 2023}
\end       {center}
\small
\vspace{10pt}
\begin{center}
\bf Total marks: 10 (One for each Question), Total Time: 15 Minutes.
\end{center}
\vspace{5pt}
\medskip\hrule
\vspace{20pt}
\large
 1. The correct way to instantiate(create object) the following Dog class is:
\vspace{20pt}
\begin{center}
\begin{lstlisting}[language=Python]
	class Dog:
	    def__init__(self, name, age):
	        self.name = name
	        self.age = age
\end{lstlisting}
\end{center}
\vspace{20pt}

(A) Dog("Rufus", 3)         (B) Dog.\_\_init\_\_("Rufus", 3)          (C)  Dog()           (D) Dog.create("Rufus", 3)

\vspace{20pt}
 2. In Python, a function within a class definition is called a:
\vspace{20pt}

(A) a method           (B) an operation           (C)  a callable       (D) a factory

\vspace{20pt}

 3. In which of the following does the Dog class correctly inherit from the Animal class?
\vspace{20pt}

(A) from Dog import Animal (B) class Dog(Animal) (C) an = Animal() (D) Dog = Animal()
\vspace{20pt}

 4.  Which of the following does not correctly create an object instance?
\vspace{20pt}

(A) obj = Cat("Jamie")    (B) cat = Cat("Jamie")   (C)       jamie = Cat()    (D)  pupper = new Cat("Jamie")

\vspace{20pt}
5. What does the following code output?
\begin{center}
\begin{lstlisting}[language=Python]
	class People():		
	    def __init__(self, name):
	    		self.name = name
	
	    def namePrint(self):
	      		print(self.name)
	
	person1 = People("Sally")
	person2 = People("Louise")
	person1.namePrint()
\end{lstlisting}
\end{center}
\vspace{20pt}

(A)  Sally         (B)  Louise        (C)     Sally Louise     (D)  person1

\vspace{20pt}

6. Which of the following is the correct way to define an initializer method?
\vspace{20pt}

(A) def \_\_init\_\_(title, author): (B) def \_\_init\_\_(self, title, author): (C) def \_\_init\_\_(): (D) \_\_init\_\_(self, title, author):

\vspace{20pt}
7. Which of the following is the parent class in the code given below?
\begin{center}
\begin{lstlisting}[language=Python]
	class A:
	
	    pass 
	
	class B(A):
	
	    pass 
	
	class C(B):
	
	    pass
   \end{lstlisting}
\end{center} 
\vspace{20pt}

(A) Class A   (B)  Class B   (C) Class C.   (D) Both A and B.

\vspace{20pt}
8.  Every class must have a constructor.
\vspace{20pt}

(A) True (B) False

\vspace{20pt}


(9). ........................: "Hiding" properties and methods of a class from the "outside world" by making these private.
\vspace{20pt}

(A) Inheritance   (B)  Polymorphism   (C) Abstraction   (D) Encapsulation.


\vspace{20pt}
10. A private method or property cannot be accessed from outside the class.
\vspace{20pt}

(A) True (B) False

\end{document}