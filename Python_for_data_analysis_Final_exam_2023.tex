\documentclass[6pt]{article}
\usepackage{ctex} 
\usepackage{amsmath,amssymb,amsfonts}
\usepackage{graphicx}
\usepackage{geometry}
\usepackage{listings}
\usepackage{parskip}
\usepackage{xcolor}
\geometry{
 a4paper,
 total={170mm,257mm},
 left=20mm,
 top=10mm,
 right=5mm,
 }

\begin{document}

\large
\begin{center}
\noindent{\bf Class-II, Course Name: Python for Data Analysis }\\
\vspace{40pt}
\bf Instructure: Dr.Zakir Ullah(zakirullah98@gmail.com).\\
\vspace{40pt}
\noindent{\bf School of Data Science, Capital University of Economics and Business}
\end       {center}

\vspace{40pt}
\begin{center}

\bf Final Exam Spring-2023, Date: 02 June 2023, Total marks: 100,  Total Time: 2Hours (18:00-20:00).
\end{center}
\vspace{5pt}
%\medskip\hrule
\vspace{40pt}


\large
\bf Instructions \\
\small
1. Please leave all personl belongings at the front of the classroom, don't  start the exam untill you are told to do so.\\
2. Talking and looking at other stusents  exam is not allowed, if you need any help please talk to your  instructure.\\
3. Please attempt all questions Questions and  encircle the correct choice.\\
\large
\vspace{20pt}




\noindent{\bf Name:---------------------------------------------------------------------------}\\
\vspace{20pt}

\noindent{\bf Student ID:--------------------------------------------------------------------- }\\
\vspace{20pt}

\noindent{\bf Date:----------------------------------------------------------------------------  }\\

\vspace{300pt}

1. 定义函数时,函数名之后的一对小括号中给出的参数是\\

\begin{flushleft}
A. 实参  \\
B. 名字参数  \\
C. 形参  \\
D. 类型参数\\
\end{flushleft}


2. 以下程序的运行结果为:\\

\begin{lstlisting}[language=Python]
class D:
    def __init__(self, w=40):
        self.__a=w+30
    def __show(self):
        print(self.__a)
    def show_private(self):
        self.__show()
obj=D(20)
obj.show_private()
\end{lstlisting}

A. 30\\
B. 20\\
\textcolor{red}{C. 50}\\
D. 70\\

3. 以下语句的运行结果为\\

\begin{lstlisting}[language=Python]
result=lambda a:pow(2, a)
result(5)
\end{lstlisting}
A. 报错\\
B. 10\\
\textcolor{red}{C. 32}\\
D. 25\\

4. Which of the following are the child classes in the code given below?\\
\begin{lstlisting}[language=Python]
clas s A:
	pass
clas s B(A) :
	pass
clas s C(B) :
	pass
\end{lstlisting}
A. A \\
B. A, B\\
\textcolor{red}{C. B, C}\\
D. A, C\\

5. 以下表达式输出结果为11的是:\\

A. print(“1+1”)\\
B. print(1+1)\\
C. print(eval(“1=1”))\\
\textcolor{red}{D. print(eval(“1”+”1”))}\\

6. 以下查看Pandas的DataFrame实例对象a的属性信息的正确方法是:\\

A. a.help\\
B. a.dtypes\\
C. a.head\\
D. a.info\\

7. 下列不是ndarray属性的是\\

A. ndim\\
B. size\\
C. shape\\
\textcolor{red}{D. reshape}\\

8. 静态方法一般用来标识\\

A. @staticmethod\\
B. staticmethod\\
C. @classmehtod\\
D. classmethod\\

9. 如果函数没有使用return语句,则函数返回的是\\

A. 0\\
		\textcolor{red}{B. None}\\
C. 任意整数\\
D. 报错!函数必须要有返回值。\\

10. a为类A的实例对象且包含一个私有属性\_\_value,在类外部能将对象a的该私有属性值修改为3的语句是:\\

A. a.\_\_value=3\\
\textcolor{red}{B. a.\_A\_\_value=3}\\
C. A.\_a\_\_value=3\\
D. A.\_\_value=3\\

11. 以下不能创建一个字典的语句是\\

A. dic1=\{\}\\
B. dic2=\{123:345\}\\
\textcolor{red}{C. dic3=\{[1, 2, 3]:’uestc’\}}\\
D. dic3=\{(1, 2, 3):’uestc’\}\\

12. 以下语句的运行结果为:\\

\begin{lstlisting}[language=Python]
x=[1, 2, 3]
x[1:]=[2]
print(x) 
\end{lstlisting}
A. [1, 2, 2]\\
\textcolor{red}{B. [1, 2]}\\
C. [1, 2, 3]\\
D. [2, 2, 2]\\

13. 下列哪种类型数据不是Python3中的数据类型\\

A. int\\
B. float\\
C. complex\\
\textcolor{red}{D. long}\\

14. 下列哪项不是Python的标准库?\\

A. statistics\\
B. os\\
C. random\\
\textcolor{red}{D. numpy}\\


15. 以下语句的运行结果为\\

\begin{lstlisting}[language=Python]
n1=[1, 2, 3]
n2=list('python')
n1+n2
\end{lstlisting}
\textcolor{red}{A. [1, 2, 3, 'p', 'y', 't', 'h', 'o', 'n']}\\
B. [1, 2, 3, ’python’]\\
C. ['1p', '2y', '3t', 'h', 'o', 'n']\\
D. 报错\\

16. 以下关于函数的哪个描述是最正确的?\\

A. 函数用于创建对象和方法\\
B. 函数可以让代码执行得更快\\
C. 函数是一段用于执行特定任务的代码\\
D. 以上说法都是正确的\\

17. print(type(16/4))的结果为\\

A. int\\
	\textcolor{red}{B. float}\\
C. str\\
D. bool\\

18.  若 a= [1, 2, 3,"python"]+[0, (1, 2, 4, 36, 8)], print(len(a)) 的结果为\\

\textcolor{red}{A. 6}\\
B. 15\\
C. 4\\
D. 11\\

19. 以下语句的输出结果为\\
\begin{lstlisting}[language=Python]
nums=set([1, 1, 2, 3, 3, 3, 4])
len(nums)
\end{lstlisting}
A. 1  \\
B. 2\\
\textcolor{red}{C. 4}\\
D. 5\\

20. 以下程序运行结果是\\

\begin{lstlisting}[language=Python]
b=[2, 4, 5, 7]
a=list(filter(lambda x:x\%2, b))
print(a)
\end{lstlisting}
A. [2, 4]\\
B. []\\
	\textcolor{red}{C. [5, 7]}\\
D. 报错\\

21. 程序设计的三种基本控制结构不包括\\

A. 递归\\
B. 循环\\
C. 条件或选择\\
D. 顺序\\

22. 以下程序的运行结果为:\\

(1, 2, 3).append(4)\\
A. (1, 2, 3, 4)\\
B. (1, 2, 3)\\
C. (4, 1, 2, 3)\\
	\textcolor{red}{D. 报错}\\

23. 以下语句的执行结果为:\\
\begin{lstlisting}[language=Python]
d = {'Name':'Li', 'Age':23, 2022:'Year'}
'Year' in d
\end{lstlisting}
A. True\\
\textcolor{red}{B. False}\\
C. None\\
D. 2022\\

24. 以下语句的运行结果为:\\

\begin{lstlisting}[language=Python]
import numpy as np
a=b=np.array([1, 2, 3, 4]).reshape(2, 2)
np.vstack((a, b)).shape
\end{lstlisting}
A. (2, 4)\\
\textcolor{red}{B. (4, 2)}\\
C. (2, 2)\\
D. (4, 4)\\

25. 下面哪些特征不是面向对象程序设计的主要特征\\

A. 封装\\
B. 函数\\
C. 多态\\
D. 继承\\

26. What is the output of the following code?\\
\begin{lstlisting}[language=Python]
def func(a, b=20, c=12):\\
print(a, b, c)\\
func(11, 33)\\
\end{lstlisting}
A. 11, 20, 12\\
B. 11, 33\\
\textcolor{red}{C. 11, 33, 12}\\
D. 11, 20, 33\\

27. 关于字典的描述,错误的是\\

A. 字典长度是可变的\\
B. 字典是键值对的集合\\
C. 字典中的键可以对应多个值信息\\
D. 字典中元素以键信息为索引访问\\

28. 关于Python全局对象和局部对象,下列说法错误的是\\

A. 全局对象不能和局部对象重名\\
B. 非局部对象不一定是全局对象\\
C. 全局对象在程序执行的全过程有效\\
D. 局部对象仅在特定代码范围中有效\\

29. 以下程序运行结果是\\

\begin{lstlisting}[language=Python]
def add_more(l): 
    l.append(66)   
mylist = [1, 33, 44] 
print(add_more(mylist))
\end{lstlisting}
A.	1, 33, 44\\
B.	1, 33, 66\\
\textcolor{red}{C.1, 33, 44, 66}\\
D.	None of the above\\

30.   若 s= \{7, 8, 9, 10, 10, 8\} ,则print(len(s*3)) 的结果是?\\

A.	18   \\
B.	6    \\
C.	12     \\
\textcolor{red}{D.	报错}\\

31. 类中定义的函数被称为.\\

A.	Class function\\
\textcolor{red}{B.	method}\\
C.	public function\\
D.	None of the above\\

32. 下列关于Python基本输入输出函数的描述,错误的是.\\

A. eval函数的参数是”3*4”的时候,返回的值是整数”12”\\
B. print函数的参数可以是一个函数,执行结果是显示函数返回的值\\
C. 当print函数输出多个对象的时候,可以用逗号分割多个对象名\\
D. 当用户输入一个整数666的时候,input函数返回的也是整数666\\


33. 以下程序运行结果是90的是\\

\textcolor{red}{A. (x:=100)-10}\\
B. x:=100-10\\
C. (x=100)-10\\
D. x==100-10\\

34. 以下语句的执行结果是\\

7+6*5**4/2-1\\
A. 4005006\\
B. 156\\
C. 906\\
\textcolor{red}{D. 1881}\\

35. 以下语句的运行结果为:\\

\begin{lstlisting}[language=Python]
def foo(x):
	if x==1:
		return 1
	else:
  		return x+foo(x-1)
foo(5)
\end{lstlisting}
\textcolor{red}{A. 15}\\
B. 10\\
C. 5\\
D. 0\\

36. Python是一种\\

A. 面向对象的编译型\\
B. 面向过程的解释型\\
C. 面向对象的解释型\\
D. 面向过程的编译型\\

37. Python在声明类的过程中定义属性时,带有的属性可以视为私有属性,但实际不是。\\

A. 2个下划线的前缀\\
B. 1个下划线的前缀\\
C. 2个下划线的后缀\\
D. 1个下划线的后缀\\

38. 以下关于pass语句的哪个描述是正确的?\\

A. Python会忽略pass语句,就像忽略注释一样\\
B. pass语句会终止当前循环\\
C. pass不做任何事情,一般用做占位语句\\
D. 以上说法都是正确的\\

39. python中下列哪种标识符代表类的私有成员\\

A. \_\_foo\_\_\\
B. \_foo\\
\textcolor{red}{C. \_\_foo}\\
D. foo\_\_\\

40. 闰年的判定条件是能被400整除,或者能被4整除但不能被100整除,正确的Python表达式是\\

\textcolor{red}{A. year\%400==0 and year\%4==0 and year\%100!=0} \\
B. (year//400==0)or(year//4==0and year//100!=0)\\
C. (year\%400==0)or(year\%4==0 and year \%100!=0) \\
D. Year//400==0 or year//4==0 and year//100!=0 \\

41. 类方法的第一个参数对应\\

A. 第一个实参\\
B. 实例本身\\
C. 类本身\\
D. 方法本身\\

42. 以下语句的运行结果为

\begin{lstlisting}[language=Python]
[y1+y2 for y1, y2 in zip([1, 2], [3, 4])]
\end{lstlisting}
\textcolor{red}{A. [4, 6]}\\
B. [3, 7]\\
C. [1, 2, 3, 4]\\
D. 报错\\

43. 命名的不定参数在传递给函数时被封装为\\

A. 列表\\
B. 集合\\
C. 元组\\
D. 字典\\

44. 下列关于pandas中DataFrame实例对象的说法,正确的是\\

A. DataFrame是二维带索引的数组,索引可自定义\\
B. DataFrame与二维ndarray类型在数据运算上方法一致\\
C. DataFrame只能表示二维数据\\
D. DataFrame由2个Series组成\\

45. 下列程序可用于查看实例对象a的成员的是\\

A. help(a)\\
B. print(a)\\
\textcolor{red}{C. dir(a)}\\
D. a?\\

46. 以下程序的运行结果为:\\

\begin{lstlisting}[language=Python]
if None: print("Hello!")
\end{lstlisting}
A. None\\
B. ‘Hello’\\
\textcolor{red}{C. 无任何输出}\\
D. 报错\\

47. 下列选项中不能用于Python对象名称的是:\\

A. 你好\\
B. \_asdf\\
\textcolor{red}{C. 001\_asdf}\\
D. asdf\\

48. 以下程序的运行结果为:\\

\begin{lstlisting}[language=Python]
class Test:
    def __init__(self):
        self.var='old'
        self.change(self.var)
    def change(sel, v):
        v='new'
obj=Test()
obj.var
\end{lstlisting}

A. 程序错误\\
B. new\\
\textcolor{red}{C. old}\\
D. 无输出\\

49. 下列语句可以创建一个[0, 10]之间,长度为16的等差数列的是\\

A. np.linspace(0, 16, 10)\\
\textcolor{red}{B. np.linspace(0,10,16)}\\
C. np.logspace(0, 10, 16)\\
D. np.randint(0, 16, 10)\\

50. 以下语句的运行结果为:\\

\begin{lstlisting}[language=Python]
i=sum=0
while i<4:
   sum+=i
   i+=1
print(sum)
\end{lstlisting}
A. 0\\
B. 10\\
\textcolor{red}{C. 6}\\
D. 以上结果均不正确\\

51. 以下程序的运行结果是\\

\begin{lstlisting}[language=Python]
a=(1, 2, ['1, 2'])
a[2].append(3)
\end{lstlisting}

A. (1, 2, ['1, 2', '3'])\\
B. \textcolor{red}{(1, 2, ['1, 2', 3])}\\
C. [1, 2, ['1, 2', 3]]\\
D. 报错\\

52. 给Numpy数组中的一个元素分配一个错误类型的值会导致错误.\\

\textcolor{red}{A. 对}\\
B. 错\\

53. 隐式继承就是指父类能够自动获得子类的全部成员。\\

A. 对\\
B. 错\\

54. 注释的作用只是增强程序的可读性,不会实际运行。\\

\textcolor{red}{A. 对}\\
B. 错\\

55. 元组是一个有序的、不可改变的集合,可以有重复的成员.\\

\textcolor{red}{A. 对}\\
B. 错\\

56. a、b均为长度相同的列表,其中每个元素均为int数值,当进行a+b操作时,会对a中每个元素都加上b中对应位置元素的值。\\

A. 对\\
\textcolor{red}{B. 错}\\

57. 对列表使用sort方法,默认是升序排列\\

\textcolor{red}{A. 对}\\
B. 错\\

58. 使用关键字参数可以进行Python函数的乱序传参。\\

A. 对\\
B. 错\\

59. Python标准库不需要使用import语句导入即可直接使用其中的所有对象。\\

A. 对\\
B. 错\\

60. 凡是用花括号(即\{\})括起来中间用逗号隔开元素的数据结构都叫字典。\\

\textcolor{red}{A. 对}\\
B. 错\\

61. Python中可以基于多个已有的类创建新类\\

A. 对\\
B. 错\\

62. 元组可用于异构数据元素的存储。\\

\textcolor{red}{A. 对}\\
B. 错\\

63. (1, 2, 3)可作为字典的键。\\
\textcolor{red}{A. 对}\\
B. 错\\

64. 函数体内部的语句在执行时,一旦执行到return时,函数就执行完毕并将结果返回。\\

A. 对\\
B. 错\\

65. 类方法不需要以self作为第一个参数。\\
A. 对\\
B. 错\\

66. 当自定义了DataFrame的行或者列索引时,就只能按照自定义索引来索引或切片。\\

A. 对\\
B. 错\\

67. Python列表的索引是从1开始的。\\

\textcolor{red}{A. 对}\\
B. 错\\

68. 在Python 3.x中可以使用中文作为对象名称。\\

A. 对\\
B. 错\\

69. Python 中的几乎所有东西都是对象,拥有属性和方法.\\

\textcolor{red}{A. 对}\\
B. 错\\

70. lambda所定义的匿名函数中可以编写多个表达式,表达式之间需使用逗号隔开。\\

A. 对\\
B. 错\\

71. not是一个身份运算符。\\

A. 对\\
B. 错\\

72. Python列表中所有的元素必须是相同的数据类型。\\

\textcolor{red}{A. 对}\\
B. 错\\

73. Python的字符串是不可变数据类型。\\

\textcolor{red}{A. 对}\\
B. 错\\

74. 由唯一元素构成的序列称之为集合set。\\

A. 对\\
B. 错\\

75. NaN是Python中的默认缺失值。\\

\textcolor{red}{A. 对}\\
B. 错\\

76. 函数内部无法定义全局对象。\\

A. 对\\
B. 错\\

77. 我们不能用一个关键词作为变量名、函数名或任何其他标识符。\\

\textcolor{red}{A. 对}\\
B. 错\\

78. NumPy数组是异质性的,可以包含不同类型的对象。\\

A. 对\\
\textcolor{red}{B. 错}\\

79. while循环中的else语体总会被执行。\\

A. 对\\
B. 错\\

80. 如安装有matplotlib包,可以使用matplotlib.plot语句来绘制图形。\\

A. 对\\
B. 错\\

81. 对于与循环语句匹配的else语句,如果循环代码从break处终止,则执行该循环的else语句。\\

A. 对\\
B. 错\\

82. Python中的for循环可以遍历任何可迭代对象。\\

A. 对\\
B. 错\\

83. 定义函数只是规定了函数会执行什么操作,但并不会真正执行;只有调用函数时才会真正去执行函数中的代码。\\

A. 对\\
B. 错\\

84. del语句可删除任何对象。\\

\textcolor{red}{A. 对}\\
B. 错\\

85. Python代码只有一种注释方式,那就是使用\#。\\

A. 对\\
B. 错\\

86. 花括号\{\}在Python中用于创建一个空集。\\

A. 对\\
B. 错\\

87. 使用@staticmethod装饰器修饰的方法称之为类方法。\\

A. 对\\
\textcolor{red}{B. 错}\\

88. import module1, module2, module3,可以一次性导入module1, module2, module3,这三个模块。\\

A. 对\\
B. 错\\

89. 当父类方法的功能不能满足需求时,可以在子类中重写父类的方法,该过程称之为方法重载。\\

\textcolor{red}{A. 对}\\
B. 错\\

90. 3+4j不是合法的Python表达式。\\

A. 对\\
B. 错\\

91. Python中类的特性能够实现利用属性来控制或调用方法。\\

A. 对\\
B. 错\\

92. Python使用缩进来体现代码之间的逻辑关系,对缩进的要求非常严格。\\

\textcolor{red}{A. 对}\\
B. 错\\

93. 可迭代对象就是迭代器。\\

A. 对\\
B. 错\\

94. 对于Python的私有成员而言,类本身和子类均可以访问它。\\

A. 对\\
\textcolor{red}{B. 错}\\

95. 全局变量不能在函数体内直接被赋值。\\

A. 对\\
B. 错\\

96. 函数必须要有return语句。\\

A. 对\\
\textcolor{red}{B. 错}\\

97. 装饰器可以用闭包函数的特性来实现。\\

A. 对\\
B. 错\\

98. 下列代码是否可以正常工作。\\
\begin{lstlisting}[language=Python]
lf=lambda x:  return x**2
lf(4)
\end{lstlisting}

A. 可以\\
\textcolor{red}{B. 不可以}\\

99. map函数将传入的函数依次作用到序列中的每一个元素,并把结果作为新的迭代器返回。\\

A. 对\\
B. 错\\

100. lambda匿名函数可以返回多个表达式的值。\\

A. 对\\
B. 错

\end{document}