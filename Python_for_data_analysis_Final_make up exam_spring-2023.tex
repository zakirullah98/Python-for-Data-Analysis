\documentclass[6pt]{article}
\usepackage{ctex} 
\usepackage{amsmath,amssymb,amsfonts}
\usepackage{graphicx}
\usepackage{geometry}
\usepackage{listings}
\usepackage{parskip}
\usepackage{xcolor}

\geometry{
 a4paper,
 total={170mm,257mm},
 left=20mm,
 top=10mm,
 right=5mm,
 }

\begin{document}

\large
\begin{center}
\noindent{\bf Class-II, Course Name: Python for Data Analysis }\\
\vspace{40pt}
\bf Instructure: Dr.Zakir Ullah(zakirullah98@gmail.com).\\
\vspace{40pt}
\noindent{\bf School of Data Science, Capital University of Economics and Business}
\end       {center}

\vspace{40pt}
\begin{center}

\bf Makeup for Final Exam Spring Semester-2023  , Date: 11 September 2023, Total marks: 100,  Total Time: 1.5Hours (14:00-14:30).
\end{center}
\vspace{5pt}
%\medskip\hrule
\vspace{40pt}


\large
\bf Instructions \\
\small
1. Please leave all personl belongings at the front of the classroom, don't  start the exam untill you are told.\\
2. Talking and looking at other stusents'  exam is not allowed, if you need any help please talk to your  instructure.\\
3. Please attempt all questions Questions and  encircle the correct choice.\\
\large
\vspace{20pt}




\noindent{\bf Name:---------------------------------------------------------------------------}\\
\vspace{20pt}

\noindent{\bf Student ID:--------------------------------------------------------------------- }\\
\vspace{20pt}

\noindent{\bf Date:----------------------------------------------------------------------------  }\\

\vspace{300pt}


1. 以下程序的运行结果为:\\

\begin{lstlisting}[language=Python]
class D:
    def __init__(self, w=40):
        self.__a=w+30
    def __show(self):
        print(self.__a)
    def show_private(self):
        self.__show()
obj=D(60)
obj.show_private()
\end{lstlisting}

A. 100\\
B. 20\\
\textcolor{red}{C. 90}\\
D. 60\\

2. 以下语句的运行结果为\\

\begin{lstlisting}[language=Python]
result=lambda a:pow(2, a)
print(result(7))
\end{lstlisting}
A. 报错\\
B. 7\\
\textcolor{red}{C. 128}\\
D. 14\\

3. 如果函数没有使用return语句,则函数返回的是\\

A. 0\\
\textcolor{red}{B. None}\\
C. 任意整数\\
D. 报错!函数必须要有返回值。\\

4. 以下语句的运行结果为:\\

\begin{lstlisting}[language=Python]
x=[11, 12, 13, 6, 7]
x[1:]=[22]
print(x) 
\end{lstlisting}
A. [11, 12, 13, 6, 7]\\
\textcolor{red}{B. [11, 22]}\\
C. [11, 12, 13]\\
D. [13, 6, 7]\\

5. 下列哪种类型数据不是Python3中的数据类型\\

A. int\\
B. float\\
C. complex\\
\textcolor{red}{D. double}\\

6. 以下语句的运行结果为\\

\begin{lstlisting}[language=Python]
n1=[0, 1, 3]
n2=list('python')
print(n1+n2)
\end{lstlisting}
\textcolor{red}{A. [0, 1, 3, 'p', 'y', 't', 'h', 'o', 'n']}\\
B. [0, 1, 3, ’python’]\\
C. ['0p', '1y', '3t', 'h', 'o', 'n']\\
D. 报错\\



7. print(type(25/5))的结果为\\

A. int\\
\textcolor{red}{B. float}\\
C. str\\
D. long\\

8.  若 l= [7, 8, 9,"hello"]+[0, (1, 2, 4, 36, 8)] , print(len(l)) 的结果为\\

\textcolor{red}{A. 6}\\
B. 14\\
C. 4\\
D. 11\\

9. 以下语句的输出结果为\\
\begin{lstlisting}[language=Python]
s=set([1, 1, 2, 3, 3, 3, 4, 4])
print(len(s))
\end{lstlisting}
A. 1  \\
B. 2\\
\textcolor{red}{C. 4}\\
D. 8\\

10. 以下程序运行结果是\\

\begin{lstlisting}[language=Python]
b=[2, 4, 5, 7]
a=list(filter(lambda x:x%2, b))
print(a)
\end{lstlisting}
A. [2, 4]\\
B. [ ]\\
\textcolor{red}{C. [5, 7]}\\
D. 报错\\



11. 以下程序的运行结果为:\\

(1, 2, 3).append(5)\\
A. (1, 2, 3, 5)\\
B. (1, 2, 3)\\
C. (5, 1, 2, 3)\\
\textcolor{red}{D. 报错}\\

12. 以下语句的执行结果为:\\
\begin{lstlisting}[language=Python]
d = {'Name':'Li', 'Age':33, 2022:'Y'}
print('Y' in d)
\end{lstlisting}
A. True\\
\textcolor{red}{B. False}\\
C. None\\
D. 2022\\

13. 以下语句的运行结果为:\\

\begin{lstlisting}[language=Python]
import numpy as np
a=b=np.array([1, 2, 3, 4]).reshape(2, 2)
np.vstack((a, b)).shape
\end{lstlisting}
A. (2, 4)\\
\textcolor{red}{B. (4, 2)}\\
C. (2, 2)\\
D. (4, 4)\\



14. What is the output of the following code?\\
\begin{lstlisting}[language=Python]
def func(a, b=20, c=12):
	print(a, b, c)
func(11, 33)
\end{lstlisting}
A. 11, 20, 12\\
B. 11, 33\\
\textcolor{red}{C. 11, 33, 12}\\
D. 11, 20, 33\\


15. 以下程序运行结果是\\

\begin{lstlisting}[language=Python]
def add_more(l): 
    l.append(66)   
mylist = [1, 33, 44] 
add_more(mylist)
print(mylist)
\end{lstlisting}
A.	66, 1, 33, 44\\
B.	1, 33, 66\\
\textcolor{red}{C.1, 33, 44, 66}\\
D.	None of the above\\

16.   若 s= \{17, 18, 19, 20, 21, 22\} ,则print(len(s*4)) 的结果是?\\

A.	24   \\
B.	6    \\
C.	10     \\
\textcolor{red}{D.	报错}\\


17. 以下语句的运行结果为:\\

\begin{lstlisting}[language=Python]
def foo(x):
	if x==1:
		return 1
	else:
  		return x+foo(x-1)
foo(6)
\end{lstlisting}
\textcolor{red}{A. 21}\\
B. 6\\
C. 5\\
D. 0\\



18. Python在声明类的过程中定义属性时,带有的属性可以视为私有属性,但实际不是。\\

A. 2个下划线的前缀\\
\textcolor{red}{B. 1个下划线的前缀}\\
C. 2个下划线的后缀\\
D. 1个下划线的后缀\\

19. 以下关于pass语句的哪个描述是正确的?\\

A. Python会忽略pass语句,就像忽略注释一样\\
B. pass语句会终止当前循环\\
\textcolor{red}{C. pass不做任何事情,一般用做占位语句}\\
D. 以上说法都是正确的\\

20. python中下列哪种标识符代表类的私有成员\\

A. \_\_foo\_\_\\
B. \_foo\\
\textcolor{red}{C. \_\_foo}\\
D. foo\_\_\\

21. 以下语句的运行结果为

\begin{lstlisting}[language=Python]
print([y1+y2 for y1, y2 in zip([11, 3], [3, 0])])
\end{lstlisting}
\textcolor{red}{A. [14, 3]}\\
B. [6, 11]\\
C. [11, 3, 3, 0]\\
D. 报错\\

22. 命名的不定参数在传递给函数时被封装为\\

A. 列表\\
B. 集合\\
C. 元组\\
\textcolor{red}{D. 字典}\\

23. 以下程序的运行结果为:\\

\begin{lstlisting}[language=Python]
if None: print("Hello!")
\end{lstlisting}
A. None\\
B. Hello!\\
\textcolor{red}{C. 无任何输出}\\
D. 报错\\

24. 下列选项中不能用于Python对象名称的是:\\

A. 你好\\
B. \_asdf\\
\textcolor{red}{C. 001\_asdf}\\
D. asdf\\



25. 下列语句可以创建一个[0, 10]之间,长度为16的等差数列的是\\

A. np.linspace(0, 16, 10)\\
\textcolor{red}{B. np.linspace(0,10,16)}\\
C. np.logspace(0, 10, 16)\\
D. np.randint(0, 16, 10)\\

26. 以下语句的运行结果为:\\

\begin{lstlisting}[language=Python]
i=sum=0
while i<6:
   sum+=i
   i+=1
print(sum)
\end{lstlisting}
A. 0\\
\textcolor{red}{B. 15}\\
C. 6\\
D. 以上结果均不正确\\

27. 给Numpy数组中的一个元素分配一个错误类型的值会导致错误.\\

\textcolor{red}{A. 对}\\
B. 错\\

28. 注释的作用只是增强程序的可读性,不会实际运行。\\

A. 错\\
\textcolor{red}{B. 对}\\


29. 元组是一个有序的、不可改变的集合,可以有重复的成员.\\

\textcolor{red}{A. 对}\\
B. 错\\

30. a, b均为长度相同的列表,其中每个元素均为int数值,当进行a+b操作时,会对a中每个元素都加上b中对应位置元素的值。\\

A. 对\\
\textcolor{red}{B. 错}\\

31. 凡是用花括号(即\{\})括起来中间用逗号隔开元素的数据结构都叫字典。\\

A. 对\\
\textcolor{red}{B. 错}\\

32. 元组可用于异构数据元素的存储。\\

\textcolor{red}{A. 对}\\
B. 错\\


33. 类方法不需要以self作为第一个参数。\\

A. 错\\
\textcolor{red}{B. 对}\\


34. Python 中的几乎所有东西都是对象,拥有属性和方法.\\

\textcolor{red}{A. 对}\\
B. 错\\

35. lambda所定义的匿名函数中可以编写多个表达式,表达式之间需使用逗号隔开。\\

A. 对\\
\textcolor{red}{B. 错}\\

36. 由唯一元素构成的序列称之为集合set。\\

A. 对\\
\textcolor{red}{B. 错}\\

37. 我们不能用一个关键词作为变量名、函数名或任何其他标识符。\\

A. 错\\
\textcolor{red}{B. 对}\\


38. NumPy数组是异质性的,可以包含不同类型的对象。\\

A. 对\\
\textcolor{red}{B. 错}\\

39. Python中的for循环可以遍历任何可迭代对象。\\

\textcolor{red}{A. 对}\\
B. 错\\

40. del语句可删除任何对象。\\

\textcolor{red}{A. 对}\\
B. 错\\


41. 使用@staticmethod装饰器修饰的方法称之为类方法。\\

\textcolor{red}{A. 错}\\
B. 对\\


42. 13+4j不是合法的Python表达式。\\

A. 对\\
\textcolor{red}{B. 错}\\

43. Python中类的特性能够实现利用属性来控制或调用方法。\\

A. 对\\
\textcolor{red}{B. 错}\\

44. Python使用缩进来体现代码之间的逻辑关系,对缩进的要求非常严格。\\

\textcolor{red}{A. 对}\\
B. 错\\

45. 可迭代对象就是迭代器。\\

A. 对\\
\textcolor{red}{B. 错}\\

46. 对于Python的私有成员而言,类本身和子类均可以访问它。\\

A. 错\\
\textcolor{red}{B. 对}\\


47. 函数必须要有return语句。\\

A. 对\\
\textcolor{red}{B. 错}\\

48. 装饰器可以用闭包函数的特性来实现。\\

\textcolor{red}{A. 对}\\
B. 错\\

49. 下列代码是否可以正常工作。\\
\begin{lstlisting}[language=Python]
a=lambda x:  return x**4
a(14)
\end{lstlisting}

\textcolor{red}{A. 不可以}\\
B. 可以\\

  
50. map函数将传入的函数依次作用到序列中的每一个元素,并把结果作为新的迭代器返回。\\

A. 错\\
\textcolor{red}{B. 对}\\


\end{document}